\subsection{Phân tích và so sánh 3 bộ dữ liệu}


\subsubsection{Sơ lược về cả 3 bộ dữ liệu}
\textbf{Breast Cancer Wisconsin (Diagnostic)}  
\begin{itemize}
	\item \textbf{Loại bài toán:} Binary Classification .  
	\item \textbf{Đặc trưng:} 30 thuộc tính.  
	\item \textbf{Kích thước:} 569 mẫu.  
	\item \textbf{Mục tiêu:} Dự đoán khối u là lành tính (Benign) hay ác tính (Malignant).
\end{itemize}

\vspace{0.3cm}

\textbf{Wine Quality}  
\begin{itemize}
	\item \textbf{Loại bài toán:} Multiclass Classification.  
	\item \textbf{Đặc trưng:} 11 thuộc tính.  
	\item \textbf{Kích thước:} 4,898 mẫu.  
	\item \textbf{Mục tiêu:} Dự đoán chất lượng rượu trong 3 nhóm: High, Standard, Low.
\end{itemize}

\vspace{0.3cm}

\textbf{....}  
\begin{itemize}
	\item \textbf{Loại bài toán:} 
	\item \textbf{Đặc trưng:} 
	\item \textbf{Kích thước:} 
	\item \textbf{Mục tiêu:} 
\end{itemize}

\subsubsection{So sánh dựa trên Classification Report}

\textbf{Hiệu suất chung của mô hình}

\begin{itemize}
	\item \textbf{Breast Cancer Wisconsin (Diagnostic):}
	\begin{itemize}
		\item Độ chính xác cao và ổn định, dao động từ \textbf{91\% đến 96\%} khi tăng tỷ lệ huấn luyện.
		\item Macro avg (\textbf{Precision}, \textbf{Recall}, \textbf{F1-Score}) luôn trên \textbf{0.90}.
	\end{itemize}
	\item \textbf{Wine Quality:}
	\begin{itemize}
		\item Độ chính xác thấp hơn, chỉ dao động từ \textbf{74\% đến 79\%}.
		\item Macro avg thấp hơn, chỉ đạt khoảng \textbf{0.52 -- 0.62} do ảnh hưởng từ mất cân bằng lớp.
	\end{itemize}
\end{itemize}



\textbf{Ảnh hưởng của class imbalance}
\begin{itemize}
	\item \textbf{Breast Cancer Wisconsin (Diagnostic):}
	\begin{itemize}
		\item Hiệu suất tăng rõ rệt khi tăng tỷ lệ huấn luyện $\rightarrow$ mô hình học tốt hơn khi có nhiều dữ liệu.
	\end{itemize}
	\item \textbf{Wine Quality:}
	\begin{itemize}
		\item Hiệu suất tăng nhẹ khi tăng tỷ lệ huấn luyện, nhưng không đáng kể $\rightarrow$ mô hình khó khái quát hóa trên bài toán đa lớp.
	\end{itemize}
\end{itemize}

\textbf{Số lớp của dữ liệu}

\begin{itemize}
	\item \textbf{Breast Cancer Wisconsin (Diagnostic):}
	\begin{itemize}
		\item Bài toán nhị phân (Binary Classification): đơn giản, ít lớp (B hoặc M) giúp mô hình phân tách hiệu quả và tính toán nhanh.
	\end{itemize}
	\item \textbf{Wine Quality:}
	\begin{itemize}
		\item Bài toán phân loại đa lớp (Multi-Class Classification): khó hơn do phải xử lý 3 lớp (High, Standard, Low), tăng độ phức tạp cho mô hình.
	\end{itemize}
\end{itemize}

\textbf{Số thuộc tính (features)}
\begin{itemize}
	\item \textbf{Breast Cancer Wisconsin (Diagnostic):}
	\begin{itemize}
		\item Có \textbf{30 thuộc tính}, đủ lớn để mô hình học tốt nhưng không gây overfitting khi tăng dữ liệu huấn luyện.
	\end{itemize}
	\item \textbf{Wine Quality:}
	\begin{itemize}
		\item Chỉ có \textbf{11 thuộc tính}, tuy nhỏ hơn nhưng dữ liệu đa lớp và mất cân bằng khiến hiệu suất bị giới hạn.
	\end{itemize}
\end{itemize}

\newpage
\textbf{Kích thước tập dữ liệu}
\begin{itemize}
	\item \textbf{Breast Cancer Wisconsin (Diagnostic):}
	\begin{itemize}
		\item Dữ liệu nhỏ (\textbf{569 mẫu}), nhưng do bài toán đơn giản, mô hình vẫn duy trì hiệu suất cao.
	\end{itemize}
	\item \textbf{Wine Quality:}
	\begin{itemize}
		\item Dữ liệu lớn hơn (\textbf{4,898 mẫu}), cung cấp nhiều thông tin nhưng không cải thiện đáng kể hiệu suất do bài toán khó khái quát hóa.
	\end{itemize}
\end{itemize}


\subsubsection{So sánh dựa trên max\_depth}

\textbf{Breast Cancer Wisconsin (Diagnostic)}
\begin{itemize}
	\item \textbf{Hiệu suất mô hình:}  
	Tập dữ liệu này có độ chính xác rất cao, ngay cả khi độ sâu của cây quyết định ở mức trung bình (\textbf{5--7}).Cho thấy:
	\begin{itemize}
		\item Mô hình học tốt các đặc trưng của dữ liệu.
		\item Tính chất của bài toán phân loại nhị phân giúp việc phân tách dữ liệu dễ dàng hơn.
	\end{itemize}
	
	\item \textbf{Số lượng đặc trưng và kích thước dữ liệu:}  
	\begin{itemize}
		\item \textbf{Số lượng đặc trưng lớn:} Với \textbf{30 thuộc tính}, mô hình cây quyết định có nhiều điều kiện phân tách để lựa chọn, giúp tăng khả năng phân loại chính xác.
		\item \textbf{Kích thước dữ liệu nhỏ:} Với chỉ \textbf{569 mẫu}, mô hình có nguy cơ bị \textbf{overfitting} khi độ sâu của cây tăng quá lớn (ví dụ: max\_depth > 7). Điều này xảy ra do mô hình có thể học quá chi tiết các đặc trưng của dữ liệu huấn luyện, dẫn đến giảm khả năng khái quát hóa trên tập dữ liệu mới.
	\end{itemize}
	
	\item \textbf{Kiểm soát overfitting:}  
	Việc chọn giá trị \textbf{max\_depth} thích hợp (từ \textbf{5--7}) là rất quan trọng để cân bằng giữa:
	\begin{itemize}
		\item \textbf{Hiệu suất cao:} Độ chính xác tốt nhất đạt \textbf{0.9561}.
		\item \textbf{Tránh overfitting:} Độ sâu lớn có thể khiến mô hình quá khớp với dữ liệu huấn luyện.
	\end{itemize}
\end{itemize}

\textbf{Wine Quality}

\begin{itemize}
	\item \textbf{Độ phức tạp của bài toán:}  
	Bài toán phân loại đa lớp (ví dụ: \textbf{Wine Quality}) có tính chất phức tạp hơn so với bài toán phân loại nhị phân, dẫn đến:
	\begin{itemize}
		\item Mô hình cần phải xử lý nhiều điều kiện phân tách hơn để phân biệt giữa các lớp.
		\item Độ chính xác của mô hình thường thấp hơn, do khó khăn trong việc học và khái quát hóa mối quan hệ phức tạp giữa các đặc trưng và nhãn lớp.
	\end{itemize}
	
	\item \textbf{Tầm quan trọng của việc chọn giá trị max\_depth:}  
	Để mô hình hoạt động hiệu quả, việc chọn giá trị \textbf{max\_depth} tối ưu là rất quan trọng:
	\begin{itemize}
		\item \textbf{Học đủ mối quan hệ phức tạp:}  
		Độ sâu phù hợp giúp mô hình học tốt các mối quan hệ phức tạp giữa các lớp trong bài toán đa lớp.
		\item \textbf{Tránh overfitting:}  
		Độ sâu quá lớn có thể dẫn đến tình trạng mô hình học quá chi tiết dữ liệu huấn luyện, làm giảm khả năng khái quát hóa trên tập dữ liệu mới.
	\end{itemize}
	
\end{itemize}


%Data set của m
\textbf{.....}